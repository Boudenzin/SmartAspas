% exemplo/main.tex
\documentclass[a4paper,12pt]{article}

% --- Codificação recomendada para pdfLaTeX ---
\usepackage[T1]{fontenc}
\usepackage[utf8]{inputenc}

% --- Idioma principal do documento (pode trocar para english/french) ---
\usepackage[brazil]{babel}

% --- Nosso pacote (depende de csquotes e já lida com autodetecção) ---
% Opções mostradas aqui apenas como exemplo; pode omitir e usar defaults
\usepackage[language=auto,activate=true,nested=true,safeenv={listings}]{smartaspas}

% --- Demais utilidades para o exemplo ---
\usepackage{amsmath}   % para \text{} no modo matemático
\usepackage{listings}  % para bloco de código sem ativar aspas automáticas
\usepackage{lipsum}    % texto de exemplo (opcional)
\usepackage{geometry}
\geometry{margin=2.5cm}

\title{Demonstração do \texttt{smartaspas}}
\author{Exemplo}
\date{\today}

\begin{document}
\maketitle

\section{Português (auto via \texttt{babel})}
Escreva naturalmente: "Aspas automáticas 'com aninhamento' funcionando".
Observe que o pacote converte para “aspas curvas” e ‘aspas simples’ conforme o idioma ativo.

\medskip
Você também pode usar os comandos explícitos do \texttt{csquotes}:
\aspas{Aspas com \texttt{\textbackslash enquote} por baixo dos panos}.
E a forma simples: \aspasSimples{aspas simples explícitas}.

\section{Inglês (mudando em tempo de execução)}
Troque o estilo em tempo de execução:
\SmartAspasSetup{language=english}

Agora escreva: "Nested 'quotes' working fine".
% volta para português no fim do documento, se desejar:
%\SmartAspasSetup{language=brazil}

\section{Francês (guillemets)}
Ative francês:
\SmartAspasSetup{language=french}

Digite normalmente: "Texte avec 'guillemets fran\c{c}ais' ici".
% Nota: espaços finos ao redor de « » são tratados por babel/polyglossia quando ativos.

% (Opcional) voltar ao pt-BR
\SmartAspasSetup{language=brazil}

\section{Controle local: ambientes}
\subsection{\texttt{smartaspasoff}}
Dentro deste ambiente, o caractere " volta a ser literal:
\begin{smartaspasoff}
Aqui "as aspas" n\~ao s\~ao convertidas.
\end{smartaspasoff}

\subsection{\texttt{smartaspason}}
Força a ativa\c{c}\~ao local (\'util se foi desligado antes):
\smartaspasOff % desliga globalmente só para demonstrar
\begin{smartaspason}
"Aqui volta a funcionar 'com aninhamento'."
\end{smartaspason}
% Saiu do ambiente: segue desligado globalmente
\smartaspasOn  % liga de novo globalmente

\section{Math mode}
No modo matemático, use \verb|\text{…}| para tipografia de texto:
\[
f(x) = \text{“função com ‘aspas’ no texto”}.
\]

\section{Ambientes de c\'odigo}
O \texttt{csquotes} já evita a conversão em ambientes verbatim/listings.
\begin{verbatim}
"Aspas" aqui ficam literais.
'Sem' conversão automática.
\end{verbatim}

Exemplo com \texttt{listings}:
\begin{lstlisting}[language=Java]
String s = "valor";
System.out.println("JSON: {\"k\": \"v\"}");
\end{lstlisting}

\section{Dicas r\'apidas}
\begin{itemize}
  \item Para documentos em pdfLaTeX, mantenha \verb|\usepackage[T1]{fontenc}| e \verb|\usepackage[utf8]{inputenc}|.
  \item Em XeLaTeX/LuaLaTeX, você pode remover \verb|inputenc| (Unicode nativo).
  \item Se algum pacote usar muito o caractere \verb|"| em chaves/opções (por exemplo, certas chaves do TikZ), rode esse trecho dentro de \verb|smartaspasoff|, ou faça:
\begin{verbatim}
\smartaspasOff
% ... trecho sensível ...
\smartaspasOn
\end{verbatim}
  \item Se usar ambientes adicionais de código, adicione em \verb|safeenv|:
\begin{verbatim}
\usepackage[safeenv={listings,tcolorboxlisting}]{smartaspas}
\end{verbatim}
\end{itemize}

\end{document}
